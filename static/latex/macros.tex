% These three definitions use \DeclarePairedDelimiter from mathtools
% Use with a * to get \left and \right sizing
\DeclarePairedDelimiter{\abs}{\lvert}{\rvert}
\DeclarePairedDelimiter{\norm}{\lVert}{\rVert}
\DeclarePairedDelimiter{\PB}{\{}{\}}{#1,#2} % Poisson brackets
\DeclarePairedDelimiterX{\innerp}[2]{\langle}{\rangle}{#1,#2}

\newcommand{\normp}[2]{{\norm*{#1}}_{#2}}

\newcommand{\Tee}{\mathsf{T}} % I use this for the transpose of a matrix

% If you have words in a subscript, they should be in Roman characters
\newcommand{\hc}{h_{\mathrm{crit}}}

% Shortcuts for some matrices, which are bold uppercase Roman letters
\newcommand{\J}{\mathbf{J}}
\newcommand{\I}{\mathbf{I}}
\newcommand{\A}{\mathbf{A}}

% Shortcuts for some vectors, which are bold lowercase Roman letters
\newcommand{\x}{\mathbf{x}}
\newcommand{\y}{\mathbf{y}}
\newcommand{\q}{\mathbf{q}}
\newcommand{\p}{\mathbf{p}}
\newcommand{\z}{\mathbf{z}}

\renewcommand{\r}{\mathbf{r}}
\newcommand{\R}{\mathbf{R}}

\DeclareMathOperator{\sech}{sech}

% First and nth ordinary and partial derivatives
% Note that the "d" in the derivative is non-italic
%\newcommand{\diff}[2]\frac{\mathrm{d}#1}{\mathrm{d}#2}
%\newcommand{\diffn}[2]\frac{\mathrm{d^{#3}}#1}{{\mathrm{d}#2}^{#3}}
%\newcommand{\diff}[2]\frac{\partial#1}{\partial#2}
%\newcommand{\diffn}[2]\frac{\mathrm{d^{#3}}#1}{{\partial#2}^{#3}}



% For use with the todonotes package. I create one of these for each author
\newcommand{\RG}[1]{\todo[inline,color=yellow]{RG: #1}}
\newcommand{\BMB}[1]{\todo[inline,color=green]{BMB: #1}}

